      \chapter{Requirement Analysis}
        \section{SOFTWARE REQUIREMENT}
Our sentiment analysis system requires Python, PHP, MySQL, TensorFlow, Flask,
Django, JavaScript, Keras which are described below;
            \subsection{Python}
Python is a high-level interpreted programming language for general-purpose
programming created by Guido van Rossum which as released in 1991. Python has
design philosophy that emphasizes code readability, notably using significant white
spaces. It provides constructs that enable clear programming in small and large scales.
The programming language features dynamic type system and automatic memory
management with support to multiple programming paradigms including Object
Oriented Imperative, Functional and Procedural. The programming language has large
comprehensive standard library.
            \subsection{PHP}
Hypertext Preprocessor (or simply PHP) is a server-side scripting language designed
for web development but also used as a general-purpose programming language. The
language is used for creating sentiment labeling system as shown in Appendix of this
report. Annotator clicks the “positive” or “negative” label which is then saved in JSON
format read by data preprocessing system.
            \subsection{MySQL}
MySQL is an open-source relational database management system (RDBMS). Its name
is a combination of "My", the name of co-founder Michael Widenius's daughter, and
"SQL", the abbreviation for Structured Query Language. In our project, we use this
database to feed into PHP system for data labeling. The database is created from
scraping Nepali Sentiment text in python.
            \subsection{Keras}
Keras is an open source neural network library written in Python which is capable of
running on top of TensorFlow, Microsoft Cognitive Toolkit, or Theano. Actual model
preparation of different models is done using Keras machine learning framework. We
5use Keras functional API to generate models as well as train the model on different
annotated data.
            \subsection{Flask}
Flask is a micro web framework written in Python. It is classified as a micro-
framework because it does not require particular tools or libraries. We use flask to
create sentiment inference engine that will be created by training the sentiment
classification model.
            \subsection{JavaScript}
JavaScript often abbreviated as JS, is a high-level, interpreted programming language.
It is a language which is also characterized as dynamic, weakly typed, prototype-based
and multi-paradigm. We use JS to both accompany in sentiment labeling system as well
as sentiment inference system which is shown in appendix part of this report.
        \section{FUNCTIONAL REQUIREMENT}
            \subsection{Dataset Labeler}
Dataset labeler is the labelling system that is used to annotate the texts as “positive”, or
“negative”.
            \subsection{Nepali News Web Scrapper}
It is the tool that extracts various Nepali text from twitter as well as various Nepali news
portals to fed into Dataset Labeler as well as vector creation process.
            \subsection{Inference System}
It is the system formed after training the model using the dataset labeled from dataset
labeler and run the prediction model in it.
        \section{NON-FUNCTIONAL REQUIREMENT}
These requirements are not needed by the system but are essential for the better
performance of sentiment engine. The points below focus on the non-functional
requirement of the system.
            \subsection{Reliability}
The system is reliable. Sentiment prediction matches 80% of the time.
            \subsection{Maintainability}
A maintainable system is created and Sentiment Analyzer Engine is able to train on
new input data and is scalable to millions of data points.
            \subsection{Performance}
The forward pass from the neural network is a fast process. For the engine, fast matrix
computation occurs.
            \subsection{Portability}
Sentiment Analyzer engine is portable and it is easy to integrate into any web
application or mobile application imaginable by the use of the REST API’s made.
        \section{FEASIBILITY STUDY}
The following points describes the feasibility of the project.
            \subsection{Economic Feasibility}
The total expenditure of the project is just computational power. The dataset and
computational power required for the project are easily available. Dataset is found from
the local news site and computational power using our own laptop along-side Google
Collaboratory cloud computing service so, the project is economically feasible.
            \subsection{Technical Feasibility}
The project is trained with lots of labeled sentiment news data and twitter tweets.
Preparing the data has its complexity like scraping from news websites and manually
labeling the dataset for sentiment classification. Creating the project will be challenging
but is feasible. In terms of availability, the dataset is easily web-scrapped from popular
news portals, social networking sites like; twitter and labeled creating a PHP web
application.
Training huge news dataset takes a lot of computation power. Which is trained in
INTEL i5 processor for some hours for about 10K data points. Training the project is
also feasible and is better with huge data points and large processing power.
            \subsection{Operational Feasibility}
The project can be operational just after training from labeled data using neural network
models. After the engine for sentiment analysis for Nepali text is created, the engine
can be operated to implement in different recommender systems and text analyzers.
Thus, the project is operationally feasible.
    
\chapter{Requirement Analysis}
\section{Software Requirement}
    Our Deepfake Detection project requires following softwares:
    \subsection*{Python} 
    Python is our primary language for this project, chosen for its versatility and simplicity. All machine learning frameworks are imported using
    Python, leveraging its dominant position in data science and machine learning. This allows us to seamlessly integrate powerful libraries like TensorFlow and PyTorch for efficient model development.

    \subsection*{Pytorch}
    Pytorch is a an open-source machine learning framework. It provides a flexible and dynamic computational graph, which allows for easy experimentation and rapid development of deep learning models. 

    \subsection*{React}
    Through React, we can create and manage the user interface components, ensuring a smooth and interactive experience. React, being a JavaScript library, allows us to efficiently build dynamic and responsive UIs, facilitating a user-centric design approach for our project.

    \subsection*{FastAPI}
    FastAPI is a Python web framework designed for rapid API development.It serves as the primary web framework for building efficient and high-performance APIs for our project.
        
\newpage
\section{Functional Requirement}
    These are specifications that describe the fundamental capabilities and behaviors a system or product must exhibit to meet the users' needs and achieve its intended purpose. 

    \subsection{Dataset Labeler}
        Dataset labeler is the labelling system that is used to annotate the images as “Real”, or “Fake”.

    \subsection{User-Friendly Dashboard}
        A user-friendly dashboard that allows user to monitor and manage the deepfake detection process.

\section{Non-Functional Requirement}
    These are the characteristics and qualities that describe how a system should behave and perform,

    \subsection{Reliability}
        The system aims for high reliability, ensuring accurate predictions with a confidence level of 97 percent.
    \subsection{Maintainability}
        The model is designed with maintainability in mind, allowing for easy updates and further training. This ensures adaptability to evolving datasets and changing input patterns, contributing to sustained efficiency and relevance over time.
    \subsection{Portability}
        The use of standardized and widely supported libraries ensures that the model can seamlessly run on various operating systems and integrate with diverse hardware configurations.
    
\newpage
\section{Feasibility Study}
    The following points describes the feasibility of the project.

    \subsection{Economic Feasibility}
        The entirety of the project's expenses is attributed solely to computational resources. The total cost sustained for the computational power utilized amounts to \$205.

    \subsection{Technical Feasibility}
        A substantial quantity of pre-labeled datasets is readily accessible online, fulfilling a pivotal necessity for this project. Leveraging the capabilities of the Google Cloud Platform (GCP), we can meet our technical requirements efficiently, harnessing the immense processing power of the A100 80GB GPU for accelerated performance.

    \subsection{Operational Feasibility}
        The operational processes, including data labeling and model training, are well-defined and can be efficiently carried out by the project team. Additionally, the project aligns with the existing technical infrastructure and capabilities, making it operationally feasible.
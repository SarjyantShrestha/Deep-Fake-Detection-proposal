      \chapter{Requirement Analysis}
        \section{SOFTWARE REQUIREMENT}
            Our Deepfake Detection project requires following softwares:
            \subsection{Python}
                Python is a general-purpose, high-level programming language. With a strong emphasis on indentation, its design philosophy prioritizes code readability. Python uses garbage collection and dynamic typing. It is compatible with various programming paradigms, such as object-oriented, functional, and structured (especially procedural). It has an extensive standard library.

            \subsection{React}
                React is a free and open-source front-end JavaScript toolkit for creating component-based user interfaces. It is also referred to as React.js or ReactJS. It is maintained by a group of independent developers and businesses as well as Meta (previously Facebook).

            \subsection{FastAPI}
                A contemporary web framework for creating RESTful Python APIs is called FastAPI. Since its initial release in 2018, its robustness, speed, and ease of use have helped it rapidly acquire favor among developers. Based on Pydantic, FastAPI serializes and deserializes data using type hints for validation. For APIs created with it, OpenAPI documentation is also automatically generated.

            \subsection{TensorFlow}
                TensorFlow is a free and open-source software library for machine learning and artificial intelligence. It can be used across a range of tasks but has a particular focus on training and inference of deep neural networks. TensorFlow was developed by the Google Brain team for internal Google use in research and production. TensorFlow can be used in a wide variety of programming languages, including Python, JavaScript, C++, and Java.

            \subsection{Keras}
                Keras is a high-level, deep learning API developed by Google for implementing neural networks. It is written in Python and is used to make the implementation of neural networks easy. It also supports multiple backend neural network computation. Keras is relatively easy to learn and work with because it provides a python frontend with a high level of abstraction while having the option of multiple back-ends for computation purposes. It supports frameworks like tensorflow.
                
        \section{FUNCTIONAL REQUIREMENT}
            These are specifications that describe the fundamental capabilities and behaviors a system or product must exhibit to meet the users' needs and achieve its intended purpose. 

            \subsection{Dataset Labeler}
                Dataset labeler is the labelling system that is used to annotate the images as “Real”, or “Fake”.

            \subsection{User-Friendly Dashboard}
                A user-friendly dashboard that allows user to monitor and manage the deepfake detection process.

        \section{NON-FUNCTIONAL REQUIREMENT}
            These are the characteristics and qualities that describe how a system should behave and perform,

            \subsection{Reliability}
                The system aims for high reliability, ensuring accurate predictions with a confidence level of 97 percent.
            \subsection{Maintainability}
                The model is designed with maintainability in mind, allowing for easy updates and further training. This ensures adaptability to evolving datasets and changing input patterns, contributing to sustained efficiency and relevance over time.
            \subsection{Portability}
                The use of standardized and widely supported libraries ensures that the model can seamlessly run on various operating systems and integrate with diverse hardware configurations.
            
        \section{FEASIBILITY STUDY}
            The following points describes the feasibility of the project.

            \subsection{Economic Feasibility}
                The total expenditure of the project is just computational power. The computational resources can be fulfilled with the help of college. Therefore, the project is economically feasible.

            \subsection{Technical Feasibility}
                Large number of already labelled datasets are easily available on the internet which is the most crucial requirement for this project. And with all the resources we have access to, this project is technically feasible.

            \subsection{Operational Feasibility}
                The operational processes, including data labeling and model training, are well-defined and can be efficiently carried out by the project team. Additionally, the project aligns with the existing technical infrastructure and capabilities, making it operationally feasible.

            

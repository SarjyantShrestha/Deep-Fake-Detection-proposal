\chapter{Literature Review}
Deepfakes, which involve sharing facial images without permission, are often carried out without the knowledge or consent of individuals such as celebrities or politicians. At first, swapping of face image was done in the photo of Abraham Lincoln(Badale et al., 2018)\cite{badale2018deep}. Yang, Li \& Lyu (2019)\cite{yang2019exposing} have proposed a model to detect deep fake using head poses inconsistency. Yadav \& Salmani (2019)\cite{yadav2019deepfake} have described the working principle of the deep fake techniques along with swapping of face images in a high precision value.

Deepfakes are generally created by techniques based on
Generative Adversarial Networks (GANs) firstly introduced
by Goodfellow et al(2014). \cite{goodfellow2014generative}. GANs consist of generators and discriminators.The generator synthesizes a fake image from the given dataset, and the discriminator evaluates the authenticity of the generated image.

%  The risks inherent in deepfakes, including defamation of character, potential harm to individuals, and the spread of fake news throughout society, highlight the importance of addressing these challenges.
 
%  Existing approaches suffer from problems such as inefficiency in deepfake image detection, high error rates, long computation times, and data access inaccuracies.
%  Our work focuses on improving efficiency through a different approach to using ResNet architecture for deepfake detection.
 Figure 1 shows a compilation of Image Deepfaketechniques'approaches,attributes and features (Heidari et al., 2023)\cite{heidari2023deepfake}.


\begin{tabularx}{\textwidth}{|X|X|X|X|X|X|}
  \hline
  \textbf{Authors} & \textbf{Main Idea} & \textbf{Advantages} & \textbf{Dataset} & \textbf{Method} \\ \hline
  Zhang Zhao, and Li (2020) & Using a binary classifier trained by a CNN & The achieved accuracy was 97\%, The AUC stood at 97.6\% & Milborrow University of Cape Town dataset & CNN \\ \hline
  Lee et al. (2021) & Suggesting an approach with an effective end-to-end false face detection pipeline & Obtaining 72.52\% AUROC, Obtaining 93.99\% accuracy on difficult low-resolution pictures & HFM dataset & GAN \\ \hline
  Guo et al. (2021) & Providing a preprocessing module named AMTEN for face image forensics & AMTENnet achieves an average accuracy of up to 98.52\%, Achieves desirable preprocessing & HEF dataset & CNN \\ \hline
  Guamera et al. (2020) & Presenting an expectation-maximization method trained to identify and extract a fingerprint & Obtaining a 93\% accuracy rate, In a real-world scenario, efficacy is demonstrated, High robustness & The FACEAPP, a dataset, and CELEBA images & Expectation-maximization + CNN \\ \hline
  Ld Yue et al. (2020) & Presenting the MCNet to utilize multi-domain spatial, frequency, and compression domain characteristics & High robustness, High accuracy & ALASKA dataset & CNN \\ \hline
\end{tabularx}
\begin{tabularx}{\textwidth}{|X|X|X|X|X|X|}
  \hline
  I Yang et al. (2021) & Using the image saliency to determine the texture depth and pixel difference between & Detection accuracy is 99.90\% & Faceforensics++ dataset & CNN + simple linear iterative clustering \\ \hline  
\end{tabularx}


\chapter{Literature Review}
\\
Deepfakes, which involve the unauthorized swapping of face images, are frequently carried out without the knowledge or consent of individuals, including celebrities and politicians. Notably, historical instances, such as the facial image swapping in a photograph of Abraham Lincoln (Badale et al., 2018), underscore the longstanding nature of this challenge. Addressing these concerns, Yang, Li \& Lyu (2019) proposed a model leveraging head pose inconsistency to detect deepfakes, enabling the creation of synthetic faces for various individuals while preserving the original facial expressions.
Jagdale \& Shah (2019) introduced the NA-VSR algorithm for super resolution, involving video conversion into frames, median filtering to remove noise, and bicubic interpolation for pixel density enhancement. Additionally, Yadav \& Salmani (2019) elucidated the working principles of deep fake techniques, emphasizing the high precision value in face image swapping.
Generative Adversarial Neural Networks (GANs) play a pivotal role in deepfake generation, comprising a generator and a discriminator. The generator synthesizes fake images from a given dataset, while the discriminator evaluates the authenticity of the generated images. The inherent risks of deepfakes, including character defamation, potential harm to individuals, and the dissemination of fake news in society, highlight the importance of addressing these challenges.
Existing approaches encounter issues such as inefficiency in detecting deepfake images, high error rates, prolonged computation times, and data access inaccuracies. This work, FF-LBPH-DBN, focuses on minimizing computational complexity while efficiently applying various metrological parameters. Table 1 presents a survey-based overview of existing approaches for detecting fake images (Vivek et al., 2018).

% \begin{table}[h]
%     \caption{Data Sources with numbers}
%     \begin{tabular}{|c|c|c|c|}
%         \hline
%         \textbf{Author} & \textbf{Name of the method}  & \textbf{Classifier}  & \textbf{Data set} \\\hline
%         Nepali News Corpus & 14364 Articles  & Vectorization & \\\hline
%         Facebook & 5021 Comments & Labeling\\\hline
%         Twitter & 2160 Tweets & Labeling\\\hline
%         Annapurnapost.com & 400 Articles Vectorization, & Labeling\\\hline
%         Nagarik News & 4481Articles & Vectorization\\\hline
%         National News & 7452 Articles & Vectorization\\\hline
%         Bsgnews.com & 1000 Articles & Vectorization, Labeling\\\hline
%         Setopati News & 1090 Articles & Vectorization\\\hline
%         NELRALEC Corpora & 798961 Sentences & POS Tagging\\\hline
%     \end{tabular}
% \end{table}
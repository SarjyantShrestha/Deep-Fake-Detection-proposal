\chapter{Literature Review}
Deepfakes, which involve sharing facial images without permission, are often carried out without the knowledge or consent of individuals such as celebrities or politicians. At first, swapping of face image was done in the photo of Abraham Lincoln(Badale et al., 2018)\cite{badale2018deep}. Yang, Li \& Lyu (2019)\cite{yang2019exposing} have proposed a model to detect deep fake using head poses inconsistency. Yadav \& Salmani (2019)\cite{yadav2019deepfake} have described the working principle of the deep fake techniques along with swapping of face images in a high precision value.

Deepfakes are generally created by techniques based on
Generative Adversarial Networks (GANs) firstly introduced
by Goodfellow et al(2014). \cite{goodfellow2014generative}. GANs consist of generators and discriminators.The generator synthesizes a fake image from the given dataset, and the discriminator evaluates the authenticity of the generated image.

Handcrafted face manipulation (HFM) images dataset and soft computing neural network models (Shallow-FakeFaceNets) with an effective facial manipulating detection process were presented by Lee et al. (2021)\cite{lee2021detecting}. The Shallow-FakeFaceNets(SFFN) model is capable of identifying fake images by focusing on altered facial landmarks, using only RGB information and no metadata. The method showed improved performance in Area Under the Receiver Operating Characteristic (AUROC), with a 3.99\% F1-score and 2.91\% AUROC for detecting handcrafted fake facial images, and 93.99\% accuracy for detecting small GAN-generated fake images.

Guo et al. (2021) \cite{guo2021fake} developed an Adaptive Manipulation Traces Extraction Network (AMTEN) for image preprocessing. This component uses a convolution layer to retrieve photo manipulation traces, with weights adjusted during backpropagation for optimization. They also created a false face detector, AMTENnet, by combining AMTEN with a Convolutional Neural Network (CNN). The manipulation traces from AMTEN are processed through the CNN to develop discriminative characteristics. Tests showed that AMTENnet achieved higher detection accuracy and generalization capabilities, outperforming other methods on the Hybrid Fake Face (HFF) dataset by 7.61\%.

Guarnera et al. (2020) \cite{guarnera2020fighting} completed a study on deepfake image analysis using an expectation-maximization technique to extract Convolutional Traces (CTs), a unique fingerprint that can identify if a photo is a deepfake and the GAN architecture that created it. The CT is resistant to attacks and independent of high-level picture concepts. The study found that a simple and fast-to-compute method could outperform more complex ones. By rotating input images to find the most significant direction, performance could be improved. The method achieved an overall classification accuracy of over 98\% on deepfakes from 10 different GAN architectures. It also performed well in a real-world setting, achieving 93\% accuracy on deepfakes created by the FACE application.

%  The risks inherent in deepfakes, including defamation of character, potential harm to individuals, and the spread of fake news throughout society, highlight the importance of addressing these challenges.
 
%  Existing approaches suffer from problems such as inefficiency in deepfake image detection, high error rates, long computation times, and data access inaccuracies.
%  Our work focuses on improving efficiency through a different approach to using ResNet architecture for deepfake detection.
There have been many approaches to detect image deepfake. Figure 1 (Heidari et al., 2023) \cite{heidari2023deepfake}  shows some of the advancements in the Image Deepfake techniques.

\begin{table}
\caption{Image Deepfake Techniques}
\begin{tabularx}{\textwidth}{|X|X|X|X|X|}
  \hline
  \textbf{Authors} & \textbf{Main Idea} & \textbf{Advantages} & \textbf{Dataset} & \textbf{Method} \\ \hline
  Zhang Zhao, and Li (2020) \cite{zhang2020novel} & Using a binary classifier trained by a CNN & The achieved accuracy was 97\%, The AUC stood at 97.6\% & Milborrow University of Cape Town dataset & CNN \\ \hline
  Lee et al. (2021) \cite{lee2021detecting}& Suggesting an approach with an effective end-to-end false face detection pipeline  that can identify fake face pictures. & Obtaining 72.52\% AUROC, Obtaining 93.99\% accuracy on difficult low-resolution pictures & HFM dataset & GAN \\ \hline
  Guo et al. (2021) \cite{guo2021fake}& Providing a preprocessing module named AMTEN for face image forensics & AMTENnet achieves an average accuracy of up to 98.52\%, Achieves desirable preprocessing & HEF dataset & CNN \\ \hline
  Guarnera et al. (2020) \cite{guarnera2020fighting} & Presenting an expectation-maximization method trained to identify and extract a fingerprint & Obtaining a 93\% accuracy rate, In a real-world scenario, efficacy is demonstrated, High robustness & The FACEAPP, a dataset, and CELEBA images & Expectation-maximization + CNN \\ \hline
  I.-J. Yu et al. (2020) \cite{yu2020manipulation} & Presenting the MCNet to utilize multi-domain spatial, frequency, and compression domain characteristics & High robustness, High accuracy & ALASKA dataset & CNN \\ \hline
\end{tabularx}
\end{table}

\newpage
\begin{tabularx}{\textwidth}{|X|X|X|X|X|}\hline
  J Yang et al. (2021) \cite{yang2021detecting} & Using the image saliency to determine the texture depth and pixel difference between actual and fake facial images.& Detection accuracy is 99.90\% & Faceforensics++ dataset & CNN + simple linear iterative clustering \\ \hline  

  Hsu et al.(2020) \cite{hung2021multi} & Presentinga an image detector comprised of an enhanced DenseNet backbone network and Slamese network architecture. & Achieving a modest level of precision and recall & CelebA & Pairwise learning \\ \hline
\end{tabularx}
% \end{table}

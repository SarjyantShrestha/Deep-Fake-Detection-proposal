    \chapter{Methodology}
       \section{SOFTWARE DEVELOPMENT APPROACH}
        Agile development is a software development approach that emphasizes incremental progress and rapid cycles. It involves releasing small increments of functionality that build upon previous versions. Thorough testing is conducted for each release to ensure software quality. Agile is often employed for time-critical applications. Although this project is not time-critical this model seems to be the most optimal and practical in our case.
        \begin{figure}[hbt!]
            \center{
                \includegraphics[width=0.75\textwidth]{./img/agile.png}
                \caption{Agile Model for Software Development}
            }
        \end{figure}
        \section{DATA COLLECTION}
        We used dataset created during a Deepfake Image Detection and Reconstruction Challenge.Two datasets of real face images were used: CelebA and FFHQ. Various Deepfake images were generated using architectures such as StarGAN, GDWCT, AttGAN, StyleGAN, and StyleGAN2. Specifically, CelebA images were manipulated using pre-trained models available on GitHub for StarGAN, GDWCT, and AttGAN. Images from StyleGAN and StyleGAN2 created through FFHQ were obtained.
        
        \begin{enumerate}
            \item CelebA\cite{7410782}: A large-scale face attributes dataset containing over 200k celebrity images with 40 labels related to facial attributes such as hair color, gender, and age. The images are in 178 x 218 JPEG format.
            
            \item FFHQ\cite{NVlabs_ffhq_dataset}: A high-quality image dataset of human faces with variations in age, ethnicity, and image background. The images are in 1024 x 1024 PNG format.
            
            \item StarGAN\cite{choi2018stargan}: Capable of performing image-to-image translations on multiple domains using a single model. CelebA images were manipulated with a pre-trained model to achieve a final resolution of 256 x 256.
            
            \item GDWCT\cite{cho2019imagetoimage}: Improves styling capability. CelebA images were manipulated with a pre-trained model to achieve a final resolution of 216 x 216.
    
            \item AttGAN\cite{8718508}: Transfers facial attributes with constraints. CelebA images were manipulated with a pre-trained model to achieve a final resolution of 256 x 256.
        
            \item StyleGAN\cite{Karras_2020_CVPR}: Transfers semantic content from a source domain to a target domain with a different style. Images were generated using FFHQ as the input dataset with a resolution of 1024 x 1024.
        
            \item StyleGAN2\cite{inproceedings}: Improves StyleGAN quality with the same task. Images were generated using FFHQ as the input dataset with a resolution of 1024 x 1024.
        \end{enumerate}

        \section{Implementation}
        Deepfake images are structured and classified into fake and real face images. The images are preprocessed and normalised. The processed image is then fed into a \acrfull*{cnn} model which is repeatedly tested and iterated to generate a model that can predict the deepfake images. The model is tested using test set to generate evaluation metrics.
        \vspace{0.5in}
        \newpage
        \begin{figure}[hbt!]
            \center{
                \includegraphics[width=0.68 \textheight]{./img/layers.png}
                \caption{Block Diagram of Proposed System}
            }
        \end{figure}

        % \section{CNN}
        % A Convolutional Neural Network (CNN) is a Deep Learning algorithm
        % that can take in an input image, assign importance (learnable weights and biases) to various aspects/objects in the image, and be able to differentiate one from the other. The pre-processing required in a CNN is much lower as compared to other classification algorithms. While in primitive methods filters are hand-engineered, with enough training, CNNs have the ability to learn these filters characteristics.The architecture of a CNN is analogous to that of the connectivity pattern of Neurons in the Human Brain and was inspired by the organization of the Visual Cortex. Individual neurons respond to stimuli only in a restricted region of the visual field known as the Receptive Field. A collection of such fields overlap to cover the entire visual area.A CNN typically has three layers: a convolutional layer, a pooling layer, and a fully connected layer.

        % Convolutional Neural Networks (CNNs) are utilized for spatial feature extraction . CNNs excel in recognizing patterns within image data by employing
        % convolutional layers and filters. The initial layers detect low-level features
        % like edges and shapes, while deeper layers abstract higher-level representations of the facial structures. The convolutional process involves
        % sliding filters across the input image, enabling the network to identify spatial hierarchies and intricate patterns.
        % \begin{figure}[hbt!]
        %         \center{
        %             \includegraphics[width=0.9\textwidth]{./img/CNN.png}
        %         }
        %         \caption{Convolutional Neural Networks}
        % \end{figure}

        % \subsection*{ResNet}
        % ResNet architecture introduced the concept called Residual Blocks. In this network, we use a technique called skip connections. The skip connection connects activations of a  layer to further layers by skipping some layers in between. This forms a residual block. Resnets are made by stacking these residual blocks together. 
        % The approach behind this network is instead of layers learning the underlying mapping, we allow the network to fit the residual mapping. So, instead of say H(x), initial mapping, let the network fit,
        % \center{F(x) := H(x) - x 
        %     which gives H(x) := F(x) + x}

        % \begin{figure}[hbt!]
        %     \center{
        %         \includegraphics[width=1\textwidth]{./img/ResNet.PNG}
        %         \caption{ResNet}
        %     }
        % \end{figure}


        \newpage
        \justifying
        \subsection{Gantt Chart}
            \begin{figure}[hbt!]
                \center{
                    \includegraphics[width=1\textwidth]{./img/GANT_CHART.jpg}
                }
                \caption{Gantt chart}
            \end{figure}


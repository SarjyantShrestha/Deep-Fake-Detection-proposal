    \chapter{Methodology}
       \section{SOFTWARE DEVELOPMENT APPROACH}
Prototype model is a software development model where instead of freezing the
requirements before design or coding can proceed, a throwaway prototype is built to
understand the requirements. The prototype are usually not complete systems and many
of the details are not built in the prototype. The goal is to provide a system with overall
functionality. In this model, we create the prototype of the actual system, update the
requirements and again rebuild the system until the final requirements are met.
        \begin{figure}[hbt!]
            \centering
                \includegraphics[width=0.75\textwidth]{./img/6.1.jpg}
                \caption{Prototype Model for Software Development}
        \end{figure}
        \section{DATA COLLECTION}
We have collected the data from popular news websites such as; ratopati.com,
ekantipur.com using web-scraping technology also from twitter tweet feed of popular
Nepali tweet pages. The data is collected programmatically from the website as
different websites have different designs and architectures. The data is mainly collected
from the opinions section of the site as the section contains more opinions to mine to
give more insightful data and making labeling easy. The API for social networking
websites were also used.
For POS tagging, the data from NELRALEC [10] The following table shows the amount
of data collected from different sources and their usage in our project;
10Table 6.1 Data Sources with numbers\\

    \begin{table}[h]
        \caption{Data Sources with numbers}
        \begin{tabular}{|c|c|c|}
            \hline
            \textbf{Source} & \textbf{Number of Data}  & \textbf{Usage} \\\hline
            Nepali News Corpus & 14364 Articles  & Vectorization\\\hline
            Facebook & 5021 Comments & Labeling\\\hline
            Twitter & 2160 Tweets & Labeling\\\hline
            Annapurnapost.com & 400 Articles Vectorization, & Labeling\\\hline
            Nagarik News & 4481Articles & Vectorization\\\hline
            National News & 7452 Articles & Vectorization\\\hline
            Bsgnews.com & 1000 Articles & Vectorization, Labeling\\\hline
            Setopati News & 1090 Articles & Vectorization\\\hline
            NELRALEC Corpora & 798961 Sentences & POS Tagging\\\hline
        \end{tabular}
    \end{table}

            \subsection{Twitter Tweet Collection}
We applied for Twitter Developer Account to get API keys to gather data from twitter.
Tweepy
[9]
which is a python tweet API library was used to make API requests and
model the response in appropriate format. We used some of the queries for the API like;
‘brb1954’, ‘hello_sarkar’, ‘thapagk’, ‘nepalitweet’ etc. From these queries the tweets
were returned which then was tokenized into sentences and modeled finally saving into
JSON format. JSON format was used because serialization and deserialization of Tweet
model as easy in JSON format than other formats like; CSV, XLS. The tweet format
for saving the data is shown in the figure below;

        \begin{figure}[hbt!]
            \centering
             \includegraphics[width=0.5\textwidth]{./img/6.2.jpg}
                \caption{Tweet in JSON Format}
        \end{figure}
        \begin{figure}[hbt!]
            \centering
             \includegraphics[width=00.5\textwidth]{./img/6.3.jpg}
                \caption{Tweet tokenization and storing flow chart}
        \end{figure}
In this way JSON formatted tweets is passed into Data Labeling System (Section 6.6)
and stored in MySQL for easy access from database rather than flat file.
            \section{DATA PREPARATION}
After the data has been collected, it was processed to make into correct format so that
the neural networks can understand both the inputs and outputs. The block diagram
(figure 6.2) shows the data preparation steps of the project.
        \begin{figure}[hbt!]
            \centering
                \includegraphics[width=0.9\textwidth]{./img/6.4.jpg}
                \caption{Data Preparation Block Diagram}
        \end{figure}

                \subsection{Tokenization}
The unlabeled and unprepared data is fed into the tokenization engine. There are two
types of tokenization; one sentence level tokenization and another word level
tokenization. Following example shows both the sentence level tokenization as well as
world level tokenization.
        \begin{figure}[hbt!]
            \centering
                \includegraphics[width=0.5\textwidth]{./img/6.5.jpg}
                \caption{Tokenization }
        \end{figure}
            \subsection{Stop Words Removal}
The tokenized Nepali words will now be preprocessed to remove stop words such as;
छ, हो etc. which has no meaning or participation in sentiment analysis. This stop word
removed tokenized data will now be fed into stemming algorithm.
        \begin{figure}[hbt!]
            \centering
                \includegraphics[width=0.5\textwidth]{./img/6.6.jpg}
                \caption{Stop Words Removal Process}
        \end{figure}                    \subsection{Stemming}
We will use snowball
[11]
rule-based stemming algorithm for removing some of the
stems of the Nepali language such as; को, का, कक etc.
        \begin{figure}[hbt!]
            \centering
                \includegraphics[width=0.5\textwidth]{./img/6.7.jpg}
                \caption{Stemming Process}
        \end{figure}
            \subsection{Word2Vec}
As neural networks require numbers as the inputs. The tokenized words need to be
converted into feature vectors i.e. list of numbers. Not any random list of number but
the related numbers. There must be a correlation between two tokenized words i.e. two
vectors must be correlated with each other. Example: In English language, king and
boy are more related than queen and girl so king and boy’s vector representation should
be near to each other than other words. Word2Vec
[12]
converts those sentences into
context linked vector representation as the name suggests.
            \subsection{Labeling System}
From the unprocessed data, we need to manually label the data points. For manually
labeling the data points we created a system from PHP programming language; reading
the data points from MySQL database and then let the user choose between positive or
negative sentiment and record it in database all hosted in internet for easy access. The
data labeling system is described in detail in section 6.6 of this report.
            \subsection{POS Tagging}
POS tagging is a piece of software that assigns parts of speech to each word and some
other token. From the word vectors and their respective POS (Parts of Speech) tags
need to be placed by the system. POS tagging gives more insight to the data making
neural network little bit easier to classify sentiment data. Some of the list of POS tags
can be seen in the table below;
    \begin{table}[hbt!]
        \centering
        \caption{Some of the POS Tags from NELRALEC Tagset}
       \begin{tabular}{|p{3cm}|p{5cm}|p{3cm}|}
            \hline
            \textbf{POS Tags} & \textbf{Description} & \textbf{Example} \\
            \hline
            NN & Common Noun &  के,टोके,टाकलम \\
            \hline
            NP & Proper Noun & राम, युबराज  \\
            \hline
            JM & Masculine Adjective & मोटो, पातलो \\
            \hline
            JF & Feminine Adjective  & दुब्ली,मोटी  \\
            \hline
             VI & Infinite Verb & गर्ुु,र्गर्ुु \\
            \hline
              ... & ... & ... \\
            \hline
        \end{tabular}
    \end{table}

The following figure below shows the POS Tagging system with example in Nepali
Language:
        \begin{figure}[hbt!]
            \centering
                \includegraphics[width=0.5\textwidth]{./img/6.8.jpg}
                \caption{POS Tagging System example taken from ANN Based POS Tagging for Nepali Text}
        \end{figure}
            \subsection{One Hot Encoder}
One Hot encoding or Binarization is the process of converting the labeled data into
binary labels. The encoding scheme of our project is shown in the table below;
Table 6.3 One Hot Encoding4
        \begin{table}[hbt!]
            \centering
            \caption{One Hot Encoding}
            \begin{tabular}{|c|c|}
                \hline
                \textbf{Labels}   & \textbf{Binarization} \\ \hline
                Positive & {[}1,0{]}   \\ \hline
                Negative & {[}0,1{]}   \\ \hline
            \end{tabular}
        \end{table}

        \section{TRAINING}
After data has been prepared the data is fed into various models described in Section
6.5.3 to 6.5.6 in this report. Various models’ evaluation was carried out after training
the model. The data will be fed into the training system or neural network as shown in
the figure below;
        \begin{figure}[hbt!]
            \centering
                \includegraphics[width=0.5\textwidth]{./img/6.9.png}
                \caption{Basic Machine Learning Training Process}
        \end{figure}
        \section{ALGORITHMS AND MODELS}
As above mentioned, various algorithms and models were developed for building the
sentiment classification engine. Following topics describes all the algorithms and
models used in the project in detail;
            \subsection{Tokenization Algorithm}
There is the use of Regular Expressions also known as RegEx for splitting the
documents into tokens. There are two types of tokenization into play namely; sentence
level tokenization and word level tokenization. Since, Nepali language sentences are
separated with special characters like purnabiram (।) or question marks (?) it is easy to
split them into sentences with the use of RegEx. The algorithm for sentence level
tokenization is given below;
Sentence Level Tokenization Algorithm
                \begin{enumerate}
                    \item Start
                    \item Input Nepali Sentences
                    \item RegEx Split Sentences /(?<=[।?!]) +/
                    \item Output split chunks of Nepali Sentences
                    \item End
                \end{enumerate}
In languages that have words separated by blank spaces, the token boundary for word-
level tokenization is the blank/white space. Nepali is one such language so word-level
tokenization in Nepali can be achieved by stripping tokens at those white spaces.
However, it is not as simple as it looks, especially when dealing with punctuations.
Some punctuation is easy to handle. We first need to replace punctuations with white
spaces. Similarly hyphen (-) which is used in linking word pairs: opposite, analogy or
similar, together. In this case hyphen is considered as the part of token itself. However,
hyphen might occur independently in such cases we need to attach the words to make
it a single token. Similarly, colon (:) and period (.) can be considered as the part of
token itself. The algorithm for word level tokenization can be seen below:
                \textit{Word Level Tokenization Algorithm}
                \begin{enumerate}
                    \item Start
                    \item Input Sentence
                    \item Replace punctuations with white spaces
                    \item Make all words with (:, ., -) symbols a single token
                    \item Split the formed sentences according to white spaces
                    \item Output the tokenized sentences
                    \item End
                \end{enumerate}
                \subsection{Stemming Algorithm}
The stemming algorithm was used in our project to reduce the redundancy imposed by
various words coming from the same stems meaning similar definition for this already
implemented algorithm in snowball was used. Snowball is a small string processing
language which was designed for creating stemming algorithms that is used in
information retrieval tasks. The language was created for creating readily available
stemming algorithms and easy information retrieval.

Snowball stemming algorithm was used in our project which is the implementation of
Shrestha, I., & Dhakal, S.S. (2016) [14] which is suffix stripping in nature. The algorithm
was created from 128 suffix rules which are executed step-by-step and in iterative
manner to eliminate inflections in Nepali Language. The stemmer was tested in 5000
Nepali words to give overall accuracy around 88.78% words. The following algorithm
18shows simplified flowchart about how the snowball stemming algorithm works for
three different types of suffixes in the word.
            \textit{Suffix Category I:}
            \begin{enumerate}
                \item Start
                \item Input Words containing category I
                \item Strip the suffix
                \item Scan for Category II and III
                \item If suffix found then go to respective stemming processes else store
                \item Stop
            \end{enumerate}
            \textit{Suffix Category II:}
            \begin{enumerate}
                \item Start
                \item Input words containing Category II
                \item Check stripping criteria
                \item If satisfied then scan for category II and III and go to respective processes else store the word
                \item Stop
            \end{enumerate}
            
            \textit{Suffix Category III:}
            \begin{enumerate}
                \item Start
                \item Input words containing category III
                \item Strip the suffix
                \item Scan for Category II and III
                \item If found then go to respective stemming processes else store
                \item Stop
            \end{enumerate}
        \subsection{Word2Vec Vectorization}
Word2Vec are a group of neural network models used to produce word embeddings.
The models are shallow two layered neural network as shown in the figure below that
are trained to reconstruct the linguistic context of words. The model takes input large
corpus of text and produces a vector space, typically a several hundred dimension but
in our case 100-dimension vector, with each unique word in the corpus being assigned
a corresponding vector in space. Word vectors are positioned in such a way that the
words share common contexts in corpus are located in close distance with each other
19in space. We can visualize the vectors in by making cluster of words and know how the
word2vec [12] model has trained.

        \begin{figure}[hbt!]
            \centering
                \includegraphics[width=0.5\textwidth]{./img/6.10.jpg}
                \caption{Word2Vec Continuous Bag of Words Model}
        \end{figure}
For training Nepali Word2Vec model we first use tokenization and stemming
procedures to remove redundancy in the corpus (table 6.1) as much as possible and the
fed it into the model for training. Our model was trained in 28787 articles for 5 epochs
in 12 core Intel i7 8 th generation processor using multiprocessing which took about 12
hours to train on. Training word2vec model is a one-time process and is used to
vectorize for training models described in sections mentioned below. The model
learned the representations of 95,818 Nepali words and the graph below shows the
distribution of number of words out of vocab.
        \begin{figure}[hbt!]
            \centering
                \includegraphics[width=0.5\textwidth]{./img/6.11.jpg}
                \caption{Distribution of Number of Words OOV per sentence}
        \end{figure}
        \subsection{Perceptron}
Perceptron is an algorithm for learning binary classifier which is also a mathematical
model of a biological neuron. While in actual neurons the dendrite receives electrical
signals from the axons of other neurons, in perceptron these electrical signals are
represented as numerical values. At the synapses, the electrical signals are modulated
in various amounts which is also modeled in the perceptron using weights. Actual
neuron fires only when total input crosses a certain threshold which is modeled using a
threshold function in our case activation function. The model maps multiple values of
input into one output either belonging to some class or not. Perception is the basic
building block of a neural network. The mathematical formula for the perceptron looks
like the following;

    $$ Y_i = F($$
Where, Y is output and I is the input.
The equation can be seen figuratively as shown below;

        \begin{figure}[hbt!]
            \centering
                \includegraphics[width=0.5\textwidth]{./img/6.12.jpg}
                \caption{Single Perceptron in a Neural }
        \end{figure}
        \subsection{LSTM RNN Model}
RNN or Recurrent Neural Network are special types of neural network that is specially
developed to learn sequential data. Since, we are trying to classify the sentiment from
sequence of tokens this neural network model should perform better than previous
dense feed forward model. Unlike, the feedforward model these neural networks can
21use their internal state to process the sequences of inputs because of which they have
been used in Language Modeling, Machine Translation, Speech Recognition and
various other tasks. But because of major two problems in RNN model called vanishing
and exploding gradients problem we use LSTM RNN model for training the classifier
as LSTM model uses forget gates to decide when to forget and remember information
for long periods of time.
            \subsubsection{LSTM-RNN}
LSTM are a special kind of RNN, capable of learning long term dependencies. They
were introduced by Hochreiter and Schmidhuber in 1997
[13]
, and were refined and
popularized by many people. They work tremendously well on large variety of
problems, and are now widely used. All recurrent neural networks have the form of a
chain of repeating modules of neural network. In standard RNNs, this repeating module
will have a very simple structure, such as a single tanh layer.
        \begin{figure}[hbt!]
            \centering
                \includegraphics[width=0.5\textwidth]{./img/6.13.jpg}
                \caption{Standard RNN Network}
        \end{figure}

The LSTM’s also have a chain like structure but repeating modules have different
structures as shown in the figure below:

        \begin{figure}[hbt!]
            \centering
                \includegraphics[width=0.5\textwidth]{./img/6.14.jpg}
                \caption{RNN-LSTM Network Architecture}
        \end{figure}
The above shown LSTM cell are good at remembering the data is because they can
choose to either remember or forger the data based on the weights of the network. This
feature of this networks helps to tackle the major problem of general RNN’s.
            \subsubsection{Classification Model}
For classification, we use two layers of LSTM RNN stacked on top of each other to
learn the task. We use the bidirectional LSTM structure for training the sentiment
classification model which is a wrapper around LSTM layers that enables bidirectional
data flow in the sequences during training process. This model was trained for 3 minutes
in Google Colab GPU Kernel for 3 epochs with the testing accuracy of 83% and model
accuracy of 80%. The model used for training and inference is shown by the figure
below alongside the hyperparameters taken to train the model;
        \begin{figure}[hbt!]
            \centering
                \includegraphics[width=0.75\textwidth]{./img/6.15.jpg}
                \caption{LSTM RNN Model}
        \end{figure}
        
        \subsection{Attention based LSTM RNN Model}
            \subsubsection{Attention Mechanism}
In the models for sequence learning, RNN architecture tends to forget the longer
sequence of data. Attention is proposed as the solution to this limitation of the
architecture basically for encoder-decoder architecture but can be used for any other
architectures. Attention is proposed as a method for both alignment and translation.
23Here alignment means which part of the input sequences are relevant to each word in
the output. In attention mechanism [8] , instead of encoding the input sequence to single
fixed context vector, the model develops a context vector that is filtered specially for
each input sequences similar to how humans approach reading longer sequence of texts.
Attention model learns to output the weights instead of statistically learning the weights
in general neural network architecture. The following figure shows the attention
mechanism in context to our project.

        \begin{figure}[hbt!]
            \centering
                \includegraphics[width=0.65\textwidth]{./img/6.16.jpg}
                \caption{Attention Model Visualization}
        \end{figure}
In the above figure we can clearly see that word सुखी has higher attention in the model
than other words meaning the given words has higher probability to affect the output of
the model than other words.
            \subsubsection{Classification Model}
For classification we use slight variation to the model discussed in section 6.5.5.2. Here
we add attention mechanism to the layer just after bidirectional LSTM’s. The model
was trained for 5 minutes in Google Colab GPU achieving validation accuracy of 80%
and model accuracy of 86% iterating over for 5 times. The model along with hyper
parameters are shown below;

        \begin{figure}[hbt!]
            \centering
                \includegraphics[width=1\textwidth]{./img/6.17.jpg}
                \caption{Attention Based LSTM RNN model}
        \end{figure}
        \subsection{POS Integrated LSTM RNN Model}
We considered the use of POS tagging system alongside LSTM RNN model to increase
the accuracy of the model by increasing the dimension that the data trains on. For this
we built separate model for POS and then integrated with LSTM model.
            \subsubsection{POS Tagging Model}
Parts-of-Speech tagging model consists of two layered Bidirectional LSTM RNN
model which takes Nepali sentence as input and outputs corresponding POS tag for
each word in the input sentence. For the training of this model, Nepali National Corpus
was used. The Nepali National Corpus (NNC) is a Nepali corpus built up 13 million
words that are lemmatized and part-of-speech tagged. The corpus was created within
the NELRALEC project funded by Asia IT & C Programme of the European
Commision. The written corpus is a collection of 500 texts of 15 different genres with
2000 words published between 1990 and 1992. The sentences were randomized and
split into training and testing with 80-20 split ratio.
The sentences were converted to vectors using the pre-trained Word2vec model of 100-
dimension matrix. The POS tags are fully hierarchial in accordance to Penn Treebank
25for English language. This model has 114 output nodes for classifying 114 tags that
were found in the NNC corpus.
The POS tagger LSTM RNN model is shown below:

        \begin{figure}[hbt!]
            \centering
                \includegraphics[width=0.75\textwidth]{./img/6.18.jpg}
                \caption{POS tagger LSTM RNN model}
        \end{figure}
This model has achieved training accuracy of 98.08\% and validation accuracy of
97.42\% after training for 8 epochs in NVIDIA 1070Ti GPU.

        \begin{figure}[hbt!]
            \centering
                \includegraphics[width=0.5\textwidth]{./img/6.19.png}
                \caption{Accuracy vs Epoch of POS tagger}
        \end{figure}

                \subsubsection{Sentiment with POS}
        
        
        \begin{figure}[hbt!]
            \centering
                \includegraphics[width=0.75\textwidth]{./img/6.20.jpg}
                \caption{Model for Sentiment with POS}
        \end{figure}

Above figure shows two-input LSTM model where “sentence_ip” is input for
vectorized text and “tag_ip” is input for corresponding POS tags. The tags after
vectorization using the Keras’ embedding layer was fed into merge layer. Add layer, a
type of merge performs scalar addition of vectors. The output from merge layer was fed
into 2 layers of Bidirectional LSTM stacked with each other and then to fully connected
layer for sentiment classification.
        \begin{figure}[hbt!]
            \centering
                \includegraphics[width=0.5\textwidth]{./img/6.21.png}
                \caption{Accuracy vs Epoch for Sentiment with POS}
        \end{figure}
            \subsubsection{Sentiment with POS and Attention}
        \begin{figure}[hbt!]
            \centering
                \includegraphics[width=0.75\textwidth]{./img/6.22.jpg}
                \caption{Model for Sentiment with POS and Attention}
        \end{figure}
The layer is similar to the model described in 6.5.7.2 with addition of attention layer
after the LSTM layers.
        \begin{figure}[h]
            \centering
                \includegraphics[width=0.5\textwidth]{./img/6.23.png}
                \caption{Accuracy vs Epoch for Sentiment with POS and Attention}
        \end{figure}
        \section{DESCRIPTION OF DATA LABELING SYSTEM}
It is the system for manually annotating data points scrapped or collected from various
sources. Here. Data points are read from MySQL database and PHP application
processes the data points according to annotator’s annotation and saves them into
MySQL database as well as in JSON format which is later serialized by Python script
for training the model. The basic block diagram of the system is shown in the figure
below and screenshot for labeling system is in appendix section of this report.

        \begin{figure}[hbt!]
            \centering
                \includegraphics[width=0.5\textwidth]{./img/6.24.jpg}
                \caption{Block Diagram of Data Labeling System}
        \end{figure}
The following table shows data processing steps for the web applications using various
RegEx filters to filter out tweets, hash-tags and various other unnecessary tokens for
data labeling.

        \begin{table}[h]
            \centering
            \caption{Data Processing in Labeling System}
           \begin{tabular}{|p{3cm}|p{4cm}|p{5cm}|}
                \hline
                \textbf{S.N.} & \textbf{RegEx} & \textbf{Description} \\
                \hline
                1. & /@[a-zA-Z0-9_]*/um & Replace twitter handles with white spaces. \\
                \hline
                2. & /[।|?]*/um & Remove full stops and question marks in Nepali Script because of our system working on sentence level classification. \\
                \hline
                 3. & /[a-zA-z!]/um & Remove all English characters (a-z and A-Z) and exclamations. \\
                \hline
                 4. & /\textbackslash.\textbackslash.\textbackslash./um & Remove trailing full stops at the end of truncated tweets. \\
                \hline
                5. & /[:] /um & Remove ‘:’ characters present in tweets. \\
                \hline
                 6. & /#/um & Remove hashtags in tweet. \\
                \hline
            \end{tabular}
        \end{table}
It is one of the most challenging part of the project and using this system we have
labeled about 5,680 number of datapoints and description of labelled data into various
sentiment classes is shown below in the bar graph.
        \begin{figure}[hbt!]
            \centering
                \includegraphics[width=0.5\textwidth]{./img/6.25.jpg}
                \caption{Positive and Negative Labels}
        \end{figure}

The screenshot of the labeling system is found in the appendix section of this report.
        \section{DESCRIPTION OF INFERENCE SYSTEM}
The system is a simple web browser-based interface used to call backend machine
learning pre-trained model. We use the best model found in our analysis to run
inference. For this we have also developed modules for preprocessing input Nepali
language that contains all the preprocessing algorithm mentioned above.
30Here text input data is taken from the input box when user types Nepali sentences and
clicks go button for running the prediction model. The request from user is sent to flask
function which will be then sent to our text processing modules responsible for
tokenizing, stemming and removing stop words. After this process the text data is sent
to embedding model changing all input words tokens into numbers. Similarly, the text
is also fed into POS tagging model to output various POS tags. The vectorized numbers
will be fed into our pre-trained machine learning models as described in section 6.5 of
this document. The model then outputs the probability that the given sentence is either
positive or negative. The output is sent to the client through the server and the user gets
to know whether the input sentence is either positive or negative.
This system aims to show the inference engine that is formed from training the model
using manually labelled datapoints. The screenshot of the inference system is shown in
the appendix section of this report.
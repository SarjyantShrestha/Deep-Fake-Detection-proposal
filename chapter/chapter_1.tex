\chapter{Introduction}
\pagenumbering{arabic}
    \section{Background Introduction}
        In recent years, the landscape of digital image manipulation has changed with the advent of deepfake technology.
        This innovative approach, based on deep learning techniques, has attracted significant attention as a means of generating images by seamlessly replacing facial features of one person with those of another.
        Titled "deepfakes" by a Reddit user in 2017, these operations often utilize advanced adversarial models such as generative adversarial networks (GANs).

        In particular, the technology has been controversially used to overlay celebrity faces onto explicit content, raising concerns about fake pornography, misinformation, financial fraud and misinformation.
        Despite the ethical challenges associated with deepfakes, it is important to recognize their positive applications in areas such as virtual reality, film editing, and production.

        The basic working principle behind deepfake involves the complex process of combining, replacing, merging,  and overlaying images.
        These operations, using deep learning and machine learning techniques, force changes to digital images and videos, highlighting both the potential benefits and ethical considerations associated with this rapidly evolving technology.


    \section{Problem Statement}
        Recent trends reveal a surge in the creation of deepfakes that manipulate the facial features of both celebrities and ordinary individuals, utilizing advanced deep-learning algorithms requiring only a single image as input. This alarming issue has escalated, resulting in the public dissemination of potentially harmful and illegal images, posing a severe threat to individual privacy and casting a shadow over public trust and safety. The growing credibility of deepfakes amplifies the risks associated with malicious use, misinformation, and reputational damage.

    \section{Objective}
        The main aim of this project is 
        \begin{itemize}
            \item to identify manipulated digital media content, particularly facial features and images using deep learning techniques.
        \end{itemize}
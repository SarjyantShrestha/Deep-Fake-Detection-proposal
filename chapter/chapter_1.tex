\chapter{Introduction}
\pagenumbering{arabic}
    \section{Background Introduction}
        In recent years, the landscape of digital image manipulation has changed with the advent of deepfake technology.
        This innovative approach, based on deep learning techniques, has attracted significant attention as a means of generating images by seamlessly replacing facial features of one person with those of another.
        Titled "deepfakes" by a Reddit user in 2017, these operations often utilize advanced adversarial models such as generative adversarial networks (GANs). \cite{st2022deep} \\\\
        In particular, the technology has been controversially used to overlay celebrity faces onto explicit content, raising concerns about fake pornography, misinformation, financial fraud and misinformation.
        Despite the ethical challenges associated with deepfakes, it is important to recognize their positive applications in areas such as virtual reality, film editing, and production. \\\\
        The basic working principle behind deepfake involves the complex process of combining, replacing, merging,  and overlaying images.
        These operations, using deep learning and machine learning techniques, force changes to digital images and videos, highlighting both the potential benefits and ethical considerations associated with this rapidly evolving technology.


    \section{Problem Definition}
        As AI technology advances, deepfake videos and images have become increasingly prevalent, with both positive and negative impacts. While they offer entertainment value, they also pose serious risks, including the misuse of images for fraudulent activities and online harassment. Recognizing the detrimental effects of this technology, many efforts are underway to develop reliable methods for detecting deepfake images and videos. Though recent studies, such as "The Face Deepfake Detection Challenge" (21st International Conference on Image Analysis and Processing) \cite{jimaging8100263}, have shown hopeful methods, there's still room to make deepfake detection systems better and able to handle more content. This project wants to fill this gap by making current methods better to improve how accurate, robust, and adaptable these systems are at spotting deepfakes.

    \pagebreak
    \section{Goals and Objectives}
        The main aim of this project is 
        \begin{itemize}
            \item To enhance existing techniques for deepfake detection, with the aim of improving accuracy and robustness of detection algorithms.
        \end{itemize}
    
    \section{Scopes and Application}
        The major applications of this project are:
        \begin{itemize}
            \item To identify and prevent the use of deepfakes in phising attacks, identity thefts and other cyber threats.
            \item To verify the authenticity of images in various contexts, such as news reporting, social media.  
        \end{itemize}
    
     \chapter{Introduction}
        \pagenumbering{arabic}
        \section{Background Introduction}
        In recent years, the landscape of digital image manipulation has undergone a transformative shift with the emergence of deepfake techniques. This innovative approach, rooted in deep learning methodologies, has gained significant traction as a means of fabricating images by seamlessly replacing facial features from one individual with those of another. Coined as "deepfakes" by a Reddit user in 2017, these manipulations often leverage advanced adversarial models, such as Generative Adversarial Networks (GANs). Notably, this technology has been controversially utilized to superimpose celebrity faces onto explicit content, raising concerns related to fake pornography, misinformation, financial fraud, and hoaxes.
        Despite the ethical challenges associated with deep fakes, it is essential to acknowledge the positive applications within fields such as virtual reality, film editing, and production. The core working principles behind deep fakes involve intricate processes of merging, replacing, combining, and superimposing images. Leveraging deep learning and machine learning techniques, these manipulations give rise to convincingly altered digital images and videos, demonstrating both the potential benefits and ethical considerations associated with this rapidly advancing technology.
        \section{Problem Statement}
        In the rapidly evolving landscape of computer and automation technologies, the realm of possibilities continues to expand. Artificial Intelligence (AI) stands as a pivotal force, driving unprecedented advancements in areas such as predictive analytics, weather forecasting, automation, and the creation of sophisticated entities like deep fakes, which encompass AI-generated videos, audios, and images. While these technological strides are undeniably transformative, the misuse and exploitation of such capabilities pose significant concerns. 

        In recent times, there's been a surge in the creation of deep fakes, where the faces of celebrities or ordinary people are manipulated using just a single image and advanced deep learning algorithms. This issue is becoming more significant, as it circulates potentially harmful and illegal images of the victims to the public.

        The rise of these deceptive practices not only threatens individual privacy but also has broader implications for public trust and safety. As deep fakes become increasingly convincing, the potential for malicious use, misinformation, and damage to reputations grows. It is crucial to address this issue head-on by developing sophisticated detection mechanisms to safeguard against the harmful consequences of manipulated images. This proposal seeks to contribute to the ongoing efforts in mitigating the risks associated with deep fakes, reinforcing the integrity of visual content in the age of advanced AI technologies.
       \section{Objective}
            The main aim of this project is:
            \begin{itemize}
                \item To identify manipulated digital media content, particularly facial features and images.
                \item To implement cutting-edge deep learning and machine learning techniques.
            \end{itemize}
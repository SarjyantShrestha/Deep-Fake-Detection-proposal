\chapter{Introduction}
\pagenumbering{arabic}
    \section{Background Introduction}
        In recent years, the landscape of digital image manipulation has changed with the advent of deepfake technology.
        This innovative approach, based on deep learning techniques, has attracted significant attention as a means of generating images by seamlessly replacing facial features of one person with those of another.
        Titled "deepfakes" by a Reddit user in 2017, these operations often utilize advanced adversarial models such as generative adversarial networks (GANs).

        In particular, the technology has been controversially used to overlay celebrity faces onto explicit content, raising concerns about fake pornography, misinformation, financial fraud and misinformation.
        Despite the ethical challenges associated with deepfakes, it is important to recognize their positive applications in areas such as virtual reality, film editing, and production.

        The basic working principle behind deepfake involves the complex process of combining, replacing, merging,  and overlaying images.
        These operations, using deep learning and machine learning techniques, force changes to digital images and videos, highlighting both the potential benefits and ethical considerations associated with this rapidly evolving technology.


    \section{Problem Statement}
        As AI technology advances, deepfake videos and images have become increasingly prevalent, with both positive and negative impacts. While they offer entertainment value, they also pose serious risks, including the misuse of images for fraudulent activities and online harassment. Recognizing the detrimental effects of this technology, many efforts are underway to develop reliable methods for detecting deepfake images and videos. Notably, "The Face Deepfake Detection Challenge" \cite{jimaging8100263} has gathered our attention, with the top-performing team (VisionLabs) achieving an accuracy of 93.61\% on classifying deepfaked and real images. Our main objective is to build upon this progress and enhance theoverall detection accuracy by a few percentage points.

    \section{Objective}
        The main aim of this project is 
        \begin{itemize}
            \item to identify manipulated digital media content, particularly facial features and images using deep learning techniques.
        \end{itemize}
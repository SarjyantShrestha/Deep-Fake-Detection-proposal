\chapter{Conclusion and Future Scope}

\section{Future Scope}
Future efforts in deepfake detection could explore complex models for enhanced detection. Comprehensive hyperparameter tuning can optimize model performance. Diversifying training data to include a wider range of deepfake variations is crucial for effective generalization. Additionally, developing computationally efficient strategies, like model compression, is essential for real-time deployment, ensuring swift response to emerging threats.

\section{Conclusion}
Detecting deepfake content has always been a challenging task, given the intricate nature of these manipulations and their departure from traditional binary classification approaches. Recognizing the effectiveness of convolutional neural networks (CNNs) in addressing such complexities, we proposed a specialized CNN-based architecture tailored for deepfake image detection in this study. Remarkably, our proposed architecture achieved an impressive accuracy rate of 93.8\% when tested across five distinct data sources containing deepfake images and two separate sources containing authentic images. \\\\
